%%%%%%%%%%%%%%%%%%%%%%%%%%%%%%%%%%%%%%%%%%%%%%%%%%%%%%%%%%%%%%%%%%%%%%%%%%%%%
 %                                                                          %
 % Tue Jul 22 13:28:10 CEST 2014                                            %
 % Copyright or (C) or Copr. Commissariat a l'Energie Atomique              %
 %                                                                          %
 % This file is part of JCHRONOSS                                           %
 %                                                                          %
 % JCHRONOSS is free software: you can redistribute it and/or modify        %
 % it under the terms of the GNU Lesser General Public License as           %
 % published by the Free Software Foundation, either version 3 of the       %
 % License, or (at your option) any later version.                          %
 %                                                                          %
 % JCHRONOSS is distributed in the hope that it will be useful,             %
 % but WITHOUT ANY WARRANTY; without even the implied warranty of           %
 % MERCHANTABILITY or FITNESS FOR A PARTICULAR PURPOSE.  See the            %
 % GNU Lesser General Public License for more details.                      %
 %                                                                          %
 % You should have received a copy of the GNU Lesser General Public License %
 % along with JCHRONOSS.  If not, see <http://www.gnu.org/licenses/>.       %
 %                                                                          %
 % Version : 1.0                                                            %
 % Author  : Julien Adam <julien.adam@cea.fr>                               %
 %                                                                          %
 %%%%%%%%%%%%%%%%%%%%%%%%%%%%%%%%%%%%%%%%%%%%%%%%%%%%%%%%%%%%%%%%%%%%%%%%%%%%

\documentclass[10pt]{article}
\usepackage[left=4cm, right=4cm]{geometry}
\usepackage[T1]{fontenc}
\usepackage[utf8]{inputenc} %encoding
\usepackage[french]{babel} %language

\title{\textbf{JCHRONOSS} \\ Moteur de validation avec ordonnanceur de travaux optimisé pour environnement massivement parallèle}
\author{Julien ADAM \and Marc PÉRACHE}
\date{\today}

\begin{document}
\maketitle
\thispagestyle{empty} % remove page number
La tendance du Calcul Haute Performance (HPC) est à l'accroissement toujours plus important de la puissance de calcul grâce aux technologies d'aujourd'hui et de demain. L'exploitation efficace de ces architectures parallèles est un véritable défi et reste un des enjeux majeurs dans de nombreux secteurs comme dans le domaine du calcul scientifique. Ainsi, de nombreux projets de recherche ont vu le jour pour tenter de dompter cette nouvelle force en mettant en place des méthodes innovantes. Dans tout projet bien structuré, un passage par une phase de validation est indispensable afin de vérifier la conformité du produit aux résultats attendus. Or, dans le contexte spécifique du calcul parallèle, cette tâche se trouve rendue difficile du fait des contraintes particulières et des architectures complexes rencontrées dans le domaine du Calcul Haute Performance. Les solutions standards peinent à se confronter à ces problèmes et à y apporter des solutions viables, notamment, à propos du passage à l'échelle. Ce que nous nous proposons d'apporter ici est un moteur de validation ayant les traits d'un véritable contrôleur qualité sachant par ailleurs interagir intelligemment avec l'environnement spécifique nécessaire au calcul parallèle. Notre méthode d'approche se base sur la prise de conscience au sein même du moteur de validation, des concepts inhérents au HPC. En tenant compte de ces problématiques, nous sommes capables de mieux apprécier les différents besoins fonctionnels d'un supercalculateur et donc de répartir la charge de travail efficacement sur celui--ci. Ainsi, il est possible d'améliorer sensiblement le rendement de la phase de validation, dont découlera à terme un gain de temps et de ressources. Le développement générique de \textit{JCHRONOSS} lui offre aussi une portabilité aisée sur la majorité des gestionnaires de travaux présents sur les machines d'aujourd'hui. Grâce à plusieurs politiques algorithmiques, nous sommes, par ailleurs, en mesure d'adapter notre comportement afin d'offrir plus de souplesse au système, et ainsi garantir la rentabilisation des ressources disponibles. De nombreuses fonctionnalités viennent compléter ces concepts fondamentaux, dans le but d'augmenter l'intégration de \textit{JCHRONOSS} à des mécanismes plus complexes. Enfin, les simplicités d'installation, de configuration et d'utilisation en font un moteur de validation à large champ d'action. En résumé, l'objectif de \textit{JCHRONOSS} est d'offrir la possibilité aux projets travaillant en contexte massivement parallèle de s'inscrire dans la lignée des pratiques énoncées par les méthodes agiles, celles--ci restant, encore aujourd'hui, gages de qualité logicielle.

\end{document}